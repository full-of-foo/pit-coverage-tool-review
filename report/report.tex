\documentclass[a4paper]{article}

\usepackage[english]{babel}
\usepackage[utf8]{inputenc}
\usepackage{amsmath}
\usepackage{graphicx}
\usepackage[colorinlistoftodos]{todonotes}
\usepackage{lipsum} %This package just generates Lorem Ipsum filler text. 
\usepackage{natbib} % Harvard style bib
\bibliographystyle{IEEEtranN}
\usepackage{hyperref} % For hyperlinks in the PDF

\title{PIT: Test Coverage Tool Review\\ \Large{CA650 Assignment 2}}

\author{Anthony Troy\\ \small{14212116}}

\date{\today}

\begin{document}
\maketitle

\renewcommand{\abstractname}{Requirements}
\begin{abstract}
Given a chosen a software testing tool, either commercial or open source, produce a report that explains: the setup and usage of the tool; a sample run of the tool using its main functionality to set up and run three tests (code examples from your chosen language, screen shots etc.); an analysis of the coverage methods available using the tool (i.e. Graph, Logic, Input Space, Syntax); and an overall assessment of the usability, coverage and robustness of the tool. For instance, such a tool might be muJava. This project is due in simple A4 format (8 \textless= pages \textless= 14), accounts for  20\% of the final module grade and must be submitted via Loop on April 22\textsuperscript{nd} before 6 \MakeUppercase{pm}.



\end{abstract}

\vspace{3.5cm}
\tableofcontents

\newpage

\section{Tool Overview}
\citep{Claus} \lipsum 


\section{Mutation Testing}
\lipsum


\section{Configuration}
\lipsum


\section{Test and Coverage Runs}
\lipsum


\section{Assessment}
\lipsum

\vspace{-7.5mm}
\renewcommand{\refname}{\section*{References}}
\bibliography{report}

\end{document}
